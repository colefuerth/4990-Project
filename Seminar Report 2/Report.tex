\documentclass[12pt]{article}

% Set font families
\usepackage{helvet}    % Helvetica
\usepackage{mathptmx}  % Times New Roman
\usepackage{hyperref}  % hyperlinks

% Set title
\title{\sffamily\fontsize{16}{18}\selectfont \textbf{SEMINAR REPORT}}
\author{}
\date{}

% extra space between paragraphs
\parskip 0.5em

% Begin document
\begin{document}

% Print title
\maketitle

% Set seminar title
{\fontfamily{ptm}\selectfont\raggedright
    \textbf{SEMINAR TITLE:} \textit{Physics-Based Visual Computing for Efficient 3D Vision and Sensing} \par
    \textbf{PRESENTER:} \textit{David B. Lindell, from University of Toronto} \par
    \textbf{DATE:} \textit{\today} \par
    \textbf{TIME:} \textit{11:00am - 12:00pm} \par
    % my email
    \textbf{SUBMITTED BY: } \textit{Cole Fuerth, 104784453} \par
}

% Rest of document goes here
% 10pt times new roman
% left justified, 1 line spacing
\fontfamily{ptm}\selectfont
\fontsize{10}{12}\selectfont
\raggedright
The \textbf{Summary + Questions} sections together should be 200-300 words. Use \textbf{10pt} Times New Roman font with \textbf{single} line spacing for all items.

% summary is 12pt veranda, bolded
\fontfamily{phv}\selectfont
\fontsize{12}{14}\selectfont
\textbf{Summary:} \par

% 10pt times new roman
\fontfamily{ptm}\selectfont
\fontsize{10}{12}\selectfont
\raggedright

% ----------------------------------------------------

Coming from University of Toronto, \href{https://davidlindell.com/}{David B. Lindell} is an Assistant Professor at the University of Toronto. His research is focused mainly around computer vision and graphics. Today's colloquium was about his research in physics-based visual computing for efficient 3D vision and sensing. \par

Building on the work initially started by Pierre Bouger in his book, "The Behaviour of Light", a Radiative Transfer Equation (RTE) was developed to model the logarithmic behaviour of light with matter.

% ----------------------------------------------------
% questions is 12pt veranda, bolded
\fontfamily{phv}\selectfont
\fontsize{12}{14}\selectfont
\textbf{Questions:} \par
% 10pt times new roman for questions
\fontfamily{ptm}\selectfont
\fontsize{10}{12}\selectfont
\raggedright
% begin questions
\begin{enumerate}
    \item How effective could this method be compared to current methods on a large scale, for instance, Tesla?
    \item What drew you to Toronto? You did your Master's at Stanford, did you start your research there and continue to Toronto?
\end{enumerate}

\end{document}
